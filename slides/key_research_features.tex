
\section{Extra: FAIMS research-specific features}

\begin{sectionframe} % Custom environment required for section slides
	\frametitle{Extra: FAIMS research-specific features}
	%\framesubtitle{Subtitle}

% 	Some section slide content

% 	This is on another line
\end{sectionframe}



%----------------------------------------------------------------------------------------
\begin{frame}[allowframebreaks]{Key research-specific features}
    \begin{itemize}
        \item Workflows and data schemata are deeply customisable.
        \item Customisation is achieved using relatively simple and human-readable XML documents separate from the ‘core’ software, supporting sharing, modification, and reuse via collaborative software development platforms like GitHub.
        \item Resulting mobile applications work offline.
        \item Automated bi-directional synchronisation using local or online server.
        \item A complete version history, with review and selective rollback, is provided for all data.
        \item Binds structured, geospatial, multimedia, and free text data in one record.
        \item A mobile GIS provides layer management and tools for the creation and editing of shapes.
        \item Connects to internal and external sensors (e.g., Bluetooth / USB devices).
        \item Externally captured multimedia (e.g., dSLR photos, audio recordings) can be connected to a record, then data contained in the associated record can be used to automatically rename connected multimedia files or write file metadata.
        \item Data validated on device and/or on server.
        \item Applications can be made multilingual using a standard and well-established localisation approach..
        \item Granular and contextualised metadata and certainty.
        \item Granular and contextualised help, including images, can be provided for all data entry fields.
        \item All data entry fields can be mapped to shared vocabularies, thesauri, or ontologies via embedded URIs, supporting Linked Open Data approaches.
        \item Data can be exported from the server in a variety of formats, or can be customised to a specific data target (e.g., JSON, relational database).
    \end{itemize}
\end{frame}
% %----------------------------------------------------------------------------------------
